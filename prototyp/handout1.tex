\documentclass[11pt]{article}



\usepackage[english,german]{babel}


\usepackage{handout}


\input{xy}
\xyoption{arrow}
\xyoption{matrix}
\xyoption{curve}
\xyoption{frame}


%\newcommand{\kommentar}[1]{[{\small\em #1}]

\newcommand{\Mist}{\textsl{Mist}}



\newcommand{\bnfdef}{::=}
\newcommand{\bnfbar}{\ensuremath{\mid}}

\newenvironment{diagram}{\begin{displaymath}}{\end{displaymath}}

\newcommand{\inputcode}[2][Code]   {
  {\small
  \mbox{}
  \newline
  \mbox{}
  \hrulefill
  \verbatiminput{#1/#2.java}
  \hrulefill}}




\renewcommand{\inputcode}[2][Code]{
  {\small
  \mbox{}
  \newline\nopagebreak{}
  \mbox{}
  \hrulefill
  \lstinputlisting{#1/#2.java}
  \hrulefill}}

\newenvironment{code}{%
%  \small\mbox{}\nopagebreak{}\mbox\hrulefill{}
  {\begin{lstlisting}{}}}{%
  {\end{lstlisting}}}


%\newenvironment{diagram}{\begin{displaymath}}{\end{displaymath}}


%%% Local Variables: 
%%% mode: latex
%%% TeX-master: t
%%% End: 



\newcommand{\state}     [1]{*++[o][F-]{#1}}
\newcommand{\initstate} [1]{ *++[o][F-]{#1}}
\newcommand{\finalstate}[1]{*++[o][F=]{#1}}
\newcommand{\inout}[2]     {#1/#2}
\newcommand{\xyinout}[2]   {{^{{#1}}_{{#2}}}}
\newcommand{\xloop}[3]   {}










\uebungfalse
\handouttrue

\ausgabe{8.~Mai 2000}
\abgabe{22.~Mail 2000}


\begin{document}

\maketitle{6}{Prototyp}


\thispagestyle{empty}

\noindent%
Die Programmieraufgabe dieses Semesters ist die Entwicklung und
Implementierung eines \emph{graphischen Werkzeugs} zur Modellierung, und
Untersuchung reaktiver, paralleler Systeme. 

Die Aufgabe wird in verschiedene Teile zerlegt werden, wobei jede der
Arbeitsgruppen f"ur einen Teil verantwortlich sein wird. Aufgaben werden
beispielsweise Teile einer Oberfl"ache sein, die graphische Eingabe,
Anbindung an andere Tools, Implementierung einer Zwischensprache, ein
Parser, Speichern der Modelle im Dateisystem und | je nach Anzahl der
Gruppen | anderes mehr.

Wie angek"undigt, zerf"allt das Praktikum in zwei Teile:
\begin{enumerate}
\item Implementieren eines \emph{Prototyps} und
\item Entwickeln und Implementieren des gesamten \textsc{Pest}.
\end{enumerate}



F"ur den ersten Teil der Aufgabe sind f"ur die 8-st"undigen Teilnehmer)
\emph{2 Wochen} vorgesehen.  Er dient unter anderem zum Aufw"armen,
allerdings ist ein lauff"ahiger Prototyp Bestandteil f"ur den 8-st"undigen
Schein, nicht alleine die gew"ahlte Teilaufgabe am Ende. Mit Aufw"armen,
beziehungsweise Prototyping ist Folgendes gemeint: die Phase dient dazu,
sich mit der etwas gr"o"seren Aufgabe vertraut zu machen. Am Ende der der
ersten Phase sollte man neben einem vorf"uhrbaren Prototyp Folgendes
erreicht haben:
\begin{itemize}
\item Beherrschen der Programmierumgebung,\footnote{Ausgenommen CVS, damit
    starten wir erst im zweiten Teil.} das ist ohnehin offensichtlich.
\item Falls notwendig, Vertrautheit im Umgang mit Java, insbesondere soll
  man nach dem ersten Teil die Verwendung von \emph{Paketen} und der
  unterschiedlichen \emph{Sichtbarkeitsstufen} zur Modularisierung und
  Strukturierung gr"o"serer Aufgaben beherrschen, denn die eigentliche
  Aufgabe wird, neben der feineren Gliederung in Klassen, in einzelne
  Pakete unterteilt werden.
\item grobes Bild von der Gesamtaufgabe (es wird dazu noch eine Besprechung
  geben).
\end{itemize}

Der erste Teil des Praktikums wird durch eine \emph{Projektbesprechung}
abgeschlossen. Das Ergebnis der Besprechung wird die Unterteilung der
Aufgabe sein.

\begin{namedaufgabe}{Prototyp}
  F"ur den \emph{Prototyp} \iffalse von \Pest{}\fi nehmen wir uns den Teil
  der Aufgabe vor, der am schnellsten zumindest sichtbare Ergebnisse
  bringt.\footnote{Das ist ein Grundgedanke des Prototypingansatzes bei der
    Softwareentwicklung.} In unserem Fall ist das der graphische Teil, also
  die Herstellung einer prototypischen \emph{graphischen Oberfl"ache}.
  Diese Aufgabe ist f"ur jeder der Gruppen die selbe.
  
  Was die Gestaltung der Oberfl"ache betrifft (Kn"opfe, Menues, etcetc.),
  sind der Phantasie keine Grenzen gesetzt. Da es sich um einen Prototyp
  handelt, kann es sich dabei teilweise um noch nicht funktionale
  Interaktionsm"oglichkeiten handeln. Aber bitte im Auge behalten, da"s
  Phantasie kein Synonym f"ur Ergonomie ist\ldots
  
  Auch wenn die Gestaltung der Oberfl"ache ist in der ersten Phase relativ
  freigestellt ist, sollen folgende Interaktionsm"oglichkeiten
  \emph{funktionsf"ahig} sein.
  \paragraph{Graphische Eingabe}
  Viele Editoren heutzutage bieten | ob zum Vorteil des Benutzers sei
  dahingestellt| die M"oglichkeit, graphisch zu arbeiten (UML, SDL, \ldots)
  Wie wollen einfache reaktive Programme als
  \emph{Zustands-"Ubergangsdiagramme} darstellen, f"ur den Prototyp stelle
  man sich eine einfache automatenartige Notation vor:

  \begin{diagram}
    \xymatrix @=5.0pc{
      \ar@{.>}[r]
      &
      \initstate{1}
      \ar[r]^{ x:= 5}
      &
      \state{2}
      \ar[r]^{ x > 6}
      &
      \finalstate{3}
      \ar@(lu,ru)[]^{ x := x+1}
      \ar@/^/[l]
      }
  \end{diagram}
  Das Diagramm ist ein unverbindlicher Vorschlag, was die Gestaltung
  betrifft, haben Sie Freiheit. 

  
  Der Teil beinhaltet das Zeichnen von
  \begin{itemize}
  \item Zust"anden
  \item "Uberg"angen
  \item Anfangszust"anden.
  \end{itemize}
  "Uberg"ange und Zust"ande sollen beschriftet sein. In dem Bild sind die
  Zust"ande mit Zahlen beschriftet, es sollen auch Identifier m"oglich
  sein. Es sollen mehrere derartige Transitionsdiagramme (z.B.
  nebeneinander) gezeichnet werden k"onnen.\footnote{eine Aufwendigere
    L"osung w"are, f"ur jedes ein einzelnes Fenster aufzumachen.} Eine
  Nichtl"osung ist, dem Benutzer ein Kritzelapplet zur Verf"ugung zu
  stellen, mit anderen Worten, die genannten graphischen Elemente sollen in
  irgendeiner Form als eigenst"andige graphische Elemente auf die
  Oberfl"ache zu bekommen sein.
  \paragraph{Speichern}
  Es soll m"oglich sein, das Ergebnis in eine vom Benutzer angegebene Datei
  zu speichern. Diese Dateien sollen zur weiteren Bearbeitung in den Editor
  eingelesen werden k"onnen.
  \paragraph{Sonstiges}
  Daneben soll Ihr L"osungsvorschlag folgende eher formale Anforderungen
  erf"ullen:
  \begin{itemize}
  \item Das Tool soll stand-alone laufen k"onnen, also nicht (nur) als ein
    Applet im Browser
  \item Es soll aus mindestens zwei Java-\emph{Paketen} bestehen.
    "Uberlegen sie sich dazu eine sinnvolle Unterteilung.
  \item "`Vervollst"andigen"' Sie die Oberfl"ache mit weiteren Vorschl"agen
    f"ur die Interaktion mit dem Werkzeug. Diese Teile brauchen nicht
    funktional sein. 
  \end{itemize}
  Wer will, kann auch folgenderma"sen vorgehen und (bei geschickter
  Unterteilung) Arbeit sparen: es ist m"oglich, da"s sich \emph{je zwei
    Arbeitsgruppen} von \emph{8st"undigen Teilnehmer} zusammenarbeiten,
  geschickterweise wird man dann jede Gruppe ein Paket programmieren
  lassen. Der Zusammenschlu"s zu 4-er Gruppen f"ur den Prototyp ist
  optional.
  
  \paragraph{Anmerkung} 
  Die genannten Aufgaben lassen sich nat"urlich auf vielf"altige Weise
  l"osen.  Eine gute L"osung bedenkt, da"s der erste Prototyp nicht alles
  sein wird, was wir machen werden. Sie k"onnten beispielsweise bedenken,
  da"s man am Ende nicht nur Zeichnen will, sondern beispielsweise auch
  selektiv l"oschen, da"s man allgemein eine Zeichnung "andern k"onnen
  will. Wie geeignet ist die L"osung in Hinblick auf die Tatsache, da"s das
  Tool nicht nur zeichnen soll, sondern beispielsweise auch die Automaten
  simulieren?
  
  F"ur diese Aufgabe sind speziell \texttt{java.awt} und auch
  \texttt{java.awt.events} aus der Javabibliothek relevant (siehe auch
  Kapitel 7 und 8 aus \cite{flanagan:javanutshell}, bzw.~Lektion 8 und 9
  aus dem Java-Praktikum des letzten Semesters).

  \paragraph{Arbeitsstil.} Gehen Sie die Entwicklung des Prototypen
  {\em systematisch\/} an. Bedenken Sie, da"s die Gruppenarbeit bei
  der Entwicklung von {\sc Pest} von Ihnen vor allem {\em
  Selbst"andigkeit und Verl"a"slichkeit\/} erfordert. Am besten
  stellen Sie sich bei der Entwicklung des Prototypen darauf ein. Dazu
  gehen Sie bitte folgendermassen vor:
  \begin{enumerate}
  \item Machen Sie einen Plan! Gliedern Sie die Aufgabe in
    Teilaufgaben und machen Sie einen Terminplan dazu. Schreiben Sie
    es auf!
  \item L"osen Sie die Teilaufgaben und erf"ullen Sie dabei den
    Terminplan. 
  \end{enumerate}
  Hier ist ein nicht verbindlicher Vorschlag zum Plan.
  \begin{description}
  \item[Milestone 1.]  Machen Sie sich u.a.\ folgende Fragen klar: Welche
    Features soll der Editor unterst"utzen? Welche Datentypen brauchen Sie
    um die graphischen Objekte intern darstellen zu k"onnen? Wie werden die
    Diagramme gespeichert gespeichert?
    
    {\bf Resultat.} Definitionen von zwei Paketen, die die Datentypen
    definieren und Methoden auflisten, die die Funktionalit"at zur
    Verf"ugung stellen. Diese Methoden sind \emph{Stubs:} Funktionen mit
    Namen, Parametern und R"uckgabetyp, aber ohne Inhalt, der die
    Funktionalit"at zur Verf"ugung stellt.
    
    \textbf{\bf Zeit.} 5 Tage
    
  \item[Milestone 2.] Programmieren Sie den Prototypen. Dazu erg"anzen
    Sie die Stubs aus dem ersten Punkt mit Code, der die
    Funktionalit"at implementiert.
    
    {\bf Resultat.} Kompilierf"ahiger Javacode des Prototypen.

    {\bf Zeit.} 5 Tage    

  \item[Milestone 3.] Testen Sie den Prototypen. 

    {\bf Resultat.} Fertiger Prototyp.

    {\bf Zeit.} 5 Tage
  \end{description}
  
  Bitte bedenken Sie, da"s Sie w"ahrend der Gruppenarbeit selbst "ahnlichen
  Plan erstellen m"ussen und da"s wir nach dem gleichen Muster vorgehen
  werden. Zum Beispiel, werden Sie aufgefordert, Stubs zur Verf"ugung zur
  stellen, die die Schnittstelle zur anderen Gruppen darstellen. Es wird
  von Ihnen erwartet, da"s Sie die Stubs rechtzeitig implementieren, damit
  noch Zeit f"ur den Test des Gesamtsystems bleibt.
\end{namedaufgabe}



\input{literatur}

\end{document}



%%%%%%%%%%%%%%%%%%%%%%%%%%%%%%%%%%%%%%%%%%%%%%%%%%%%%%%%%%



