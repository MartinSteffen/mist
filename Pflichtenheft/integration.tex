\section{Graphische Benutzerschnittstelle}
\label{sec:gui}

\Mist{} besteht aus verschiedenen Komponenten, die ihrerseits mit dem
Benutzer interagie ren .

Es soll eine "ubergeordnete Schnittstelle geben, die folgende Aufgaben
bew"altigt:


\begin{itemize}
\item\textbf{Start:} Beim Start einer \Mist-Session erschheint ein Fenster,
  von wo aus es m"oglich ist, verschiedene Komponenten des Systems
  aufzurufen.
\item\textbf{Abh"angigkeitsverwaltung:} Eine Simulation kann erst dann
  aufgerufen werden, wenn das Programm syntaktisch korrekt.  Das gleiche
  gilt f"ur die Codegenerierung. Die Aufgabe besteht darin, eine Definition
  der Abh"angigkeiten zwischen den Komponen ten festzulegen und sie im Tool
  zu implemen tieren.
\item\textbf{Session verw altung} Es soll m"oglich sein, eine Session (ge
  "offnete Fenster, geladene Dateien, gew "ahlte Optionen) zu speichern.
  Eine gespeicherte Session sollte wieder hergestellt werden k"onnen.
\end{itemize}

Die Benutzeroberfl"ache \emph{integriert} alle anderen Komponenten, aus
diesem Grund ist hier besonders auf die \emph{Konsistenz} bzw.\ Verletzung
der Konsistenz zu achten.



%%% Local Variables: 
%%% mode: latex
%%% TeX-master: "handout1"
%%% End: 
