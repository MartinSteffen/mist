
\section{Parser}
\label{sec:parser}


\team{Andreas Scott, Alexander Hegener}




Es soll eine nicht-graphische einfache, imperative Sprache als
Eingabesprache erlaubt sein.  Die Sprache soll in \Mist{} so unterst�tzt
werden, da� man textuelle Spezifikationen eingeben kann, ohne da� man auf
die graphische Darstellung verzichten mu�.  Die graphische Darstellung der
Zust�nde wird von \Mist{} berechnet.


Ein Vorschlag f�r eine konkrete Syntax findet sich in
Abschnitt~\ref{sec:KonkreteSyntax}. Im ersten Schritt der Transformation
(in diesem Modul) wird das textuelle Programm geparst und als abstrakter
Syntaxbaum (ohne graphische Platzierung) dargestellt.


Die Implementierung wird \textsl{JLex} verwenden, welcher auf 
\texttt{\home{unix01}/bin/} installiert werden wird.

\subsubsection*{Schnittstelle}
Mit der Gui. Es wird eine Methode \texttt{parse\_file} zur Verf�gung
gestellt; Parameter: ein String, welcher die Datei bezeichnet, die das
Programm enth"alt.  Die Dateien sollen als Standard-extension
\texttt{.mist}-besitzen. Der Parser kann die Ausnahme
\texttt{Parser\_Exception} werfen. W"unschenswert ist, wenn der Parser
zumindest die Zeilennummer des Fehlers in der Ausnahme zur"uckgibt.





%%% Local Variables: 
%%% mode: latex
%%% TeX-master: "handout1"
%%% End: 
