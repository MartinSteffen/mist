\section{Editor}
\label{sec:editor}

�hnlich dem Prototyp soll eine \emph{Editor} f�r die Transitionssysteme
implementiert werden. Er soll folgende Eigenschaften besitzen:

\begin{itemize}
\item\textbf{Aufbau:} Es soll m�glich sein, ein Transitionssystem aus
  \emph{Schablonen} von Zust�nden, Transitionen etc., aufbauen zu k�nnen.
  Der Initialzustand soll jeweils speziell kenntlich sein.
\item \textbf{Speichern und Laden}
  Die Systeme sollen gespeichert und geladen werden k�nnen.
\item \textbf{Selektieren:} Einzelne Komponenten sollen selektiert werden
  k�nnen. Das dient zur Vorbereitung weiterer Aktionen.
\item \textbf{L�schen \& Kopieren:} Es soll m�glich sein, selektierte
  Komponenten zu entfernen und zu kopieren.
\item \textbf{Highlight}: der Editor soll eine Highlightfunktion zur
  Verf�gung stellen.  Es soll m�glich sein, Methodenaufrufe bestimmte
  Zust�nde und oder Transitionen zu highlighten.
\end{itemize}




%%% Local Variables: 
%%% mode: latex
%%% TeX-master: "handout1"
%%% End: 
