\section{Editor}
\label{sec:editor}

\team{Finn Jacobs und  Alexander Eckloff}


�hnlich dem Prototyp soll ein \emph{Editor} f�r die Transitionssysteme
implementiert werden. Er soll folgende Eigenschaften besitzen:

\begin{itemize}
\item\textbf{Aufbau:} Es soll m�glich sein, ein Transitionssystem aus
  \emph{Schablonen} von Zust�nden, Transitionen etc., aufbauen zu k�nnen.
  Der Initialzustand soll jeweils speziell kenntlich sein.
\item \textbf{Speichern und Laden:}
  Die Systeme sollen gespeichert und geladen werden k�nnen.
\item \textbf{Selektieren:} Einzelne Komponenten sollen selektiert werden
  k�nnen. Das dient zur Vorbereitung weiterer Aktionen.
\item \textbf{L�schen \& Kopieren:} Es soll m�glich sein, selektierte
  Komponenten zu entfernen und zu kopieren.
\item \textbf{Highlight}: der Editor soll eine Highlightfunktion zur
  Verf�gung stellen.  Es soll m�glich sein, Methodenaufrufe, bestimmte
  Zust�nde und/oder Transitionen zu highlighten.
\end{itemize}







\subsubsection*{Designentscheidungen der Editorgruppe/Gui-Gruppe/Graphplatzierungsgruppe:}
\begin{itemize}
\item Es wird dem Editor ein \emph{gesamtes Programm} (Klasse
  \texttt{absynt.Program}) �bergeben, welches der Editor anzuzeigen
  hat.\footnote{In einer fr�heren Version des Pflichtenheftes (siehe
    Version 3) wurde dem Editor nur einzelne Prozesse �bergeben. Um die
    Schnittstellen nicht zu verkomplizieren, wurde die Entscheidung
    zur�ckgenommen}.
\item Es wird mit \emph{Swing} gearbeit, um ein einheitliches
  \textit{Schau-und-f�hl} zu bekommen. Falls es Probleme mit der Effizienz
  gibt, wird es nicht schwer sein, auf AWT umzusatteln.
\item Koordinaten werden relativ genommen, und zwar in Flie�kommawerten
  \textbf{zwischen $0$ und $1$}. Dies sind die Werte die f�r Zust�nde die
  Position relativ zur Umgebenden Fl�che bestimmen.
\item Speichern und laden wird von der Gui �bernommen.???
\item Schnittstellen: Gui ruft Editor auf. 
\end{itemize}



\paragraph{Tempor�re Vereinbarung}
Der Parser (Abschnitt~\ref{sec:parser}) soll dem Editor ein Methode/Klasse
zur Verf�gung stellen, die es erlaubt, Expresssions zu parsen (f�r die
Transitionen).  Solange diese noch nicht bereitgestellt ist, werden diese
F�lle als \texttt{absynt.Constval} eingegeben (d.h., ein m�glicherweise
eingegebener String wird ignoriert, da der Konstruktor
\texttt{absynt.Constval} nur boolsche Werte oder Integerwerte akzeptiert.

\subsubsection*{Schnittstelle}

Mit der Gui (Abschnitt~\ref{sec:gui}). Desweiterern mit dem Simulator
(Abschnutt~\ref{sec:simulator}). Es jedoch ist noch zu diskutieren, ob der
Simulator sich \emph{direkt} an den Editor werdet oder ob dies
\emph{vermittels} der Gui passiert, da ein Editor nur due �bersicht �ber
einen Prozess hat.

Auf jeden Fall: eine Methode \texttt{highlight\_state}, als �bergabe
entweder
\begin{itemize}
\item der Bezeichner des Zustandes, oder
\item der Zustand als Objekt.
\end{itemize}
Die Wahl mu� mit dem Simulator oder der Gui vereinbart werden, abh�ngig
davon, wer die Methode aufruft.

Eine \emph{wichtige} Schnittstelle (wie bei allen) ist die abstrakte
Syntax. Um das Zeichnen zu unterst�tzen, wurde in die abstrakte Syntax
\emph{Koordinaten} mit aufgenommen.


%%% Local Variables: 
%%% mode: latex
%%% TeX-master: "handout1"
%%% End: 
