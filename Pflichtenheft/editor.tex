\section{Editor}
\label{sec:editor}

\team{Finn Jacobs und  Alexander Eckloff}


�hnlich dem Prototyp soll ein \emph{Editor} f�r die Transitionssysteme
implementiert werden. Er soll folgende Eigenschaften besitzen:

\begin{itemize}
\item\textbf{Aufbau:} Es soll m�glich sein, ein Transitionssystem aus
  \emph{Schablonen} von Zust�nden, Transitionen etc., aufbauen zu k�nnen.
  Der Initialzustand soll jeweils speziell kenntlich sein.
\item \textbf{Speichern und Laden:}
  Die Systeme sollen gespeichert und geladen werden k�nnen.
\item \textbf{Selektieren:} Einzelne Komponenten sollen selektiert werden
  k�nnen. Das dient zur Vorbereitung weiterer Aktionen.
\item \textbf{L�schen \& Kopieren:} Es soll m�glich sein, selektierte
  Komponenten zu entfernen und zu kopieren.
\item \textbf{Highlight}: der Editor soll eine Highlightfunktion zur
  Verf�gung stellen.  Es soll m�glich sein, Methodenaufrufe, bestimmte
  Zust�nde und/oder Transitionen zu highlighten.
\end{itemize}



- Speichern und laden wird von der Gui �bernommen.
- Toolkit: Swing

- Schnittstellen: Gui ruft Editor auf. 
- Highlightfunktion: 
- 




\subsubsection*{Designentscheidungen der Editorgruppe/Gui-Gruppe/Graphplatzierungsgruppe:}
\begin{itemize}
\item Es wird \emph{ein} Fenster pro Prozess gezeichnet, mit anderen
  Worten, das komplette Programm (bestehend aus mehreren Prozessen, bleibt
  bei der Gui).
\item Es wird vorl�ufig mit Swing gearbeit, um ein einheitliches
  \textit{schau-und-f�hl} zu bekommen. Falls es Probleme mit der Effizienz
  gibt, wird es nicht schwer sein, auf AWT umzusatteln.\footnote{Bemerkung
    (ms): glaube ich nicht; selbst wenn es stimmt, am Ende des Projektes
    wird niemand schnell irgendwohin umsatteln wollen.}
\item Koordinaten werden relativ genommen, und zwar in Flie�kommawerten
  \textbf{zwischen $0$ und $1$}. Dies sind die Werte die f�r Zust�nde die
  Position relativ zur Umgebenden Fl�che bestimmen.
\end{itemize}





\subsubsection*{Schnittstelle}

Mit der Gui (Abschnitt~\ref{sec:gui}). Desweiterern mit dem Simulator
(Abschnutt~\ref{sec:simulator}). Es jedoch ist noch zu diskutieren, ob der
Simulator sich \emph{direkt} an den Editor werdet oder ob dies
\emph{vermittels} der Gui passiert, da ein Editor nur due �bersicht �ber
einen Prozess hat.

Auf jeden Fall: eine Methode \texttt{highlight\_state}, als �bergabe
entweder
\begin{itemize}
\item der Bezeichner des Zustandes, oder
\item der Zustand als Objekt.
\end{itemize}
Die Wahl mu� mit dem Simulator oder der Gui vereinbart werden, abh�ngig
davon, wer die Methode aufruft.

Eine \emph{wichtige} Schnittstelle (wie bei allen) ist die abstrakte
Syntax. \textbf{Noch ungekl�rt:} sollen die Graphischen Koordinaten in die
abstakte Syntax mit aufgenommen werden, oder sollen sie davon getrennt
gehalten werden (durch eine geeignete, um Graphik-Information erweiterte
Datenstrukturen.

%%% Local Variables: 
%%% mode: latex
%%% TeX-master: "handout1"
%%% End: 
