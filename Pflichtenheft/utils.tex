\section{Hilfsprogramme}
\label{sec:utils}

Verschiedene Programme, die keinem anderen Paket zugeteilt sind und
mehreren Paketen n"utzen.


\subsection{Pretty-Printer}
\label{sec:prettyprinter}

\team{Oliver Kraus, Holger Labenda}

Ein einfacher Pretty-Printer mit tabuliertem ascii-Output, er soll vor
allem zu Diagnosezwecken dienen.


\subsubsection*{Schnittstelle}
Jeder darf ihn benutzen, er dient haupts�chlich zur Diagnose. Die Einzige
schnittstelle die z�hlt ist, da� er abstrakte Syntax ausgeben k�nnen mu�.
Die Schnittstelle ist bereits teilweise implementiert (zur Verwendung siehe
\texttt{utils.PpExample}). Es werden neben der \texttt{print}-Funktion f�r
ganze Programme gleichlautende Methoden f�r andere syntaktische Konstrukte
zur Verf�gung gestellt (\texttt{public}), damit man auch von au�en
Teilprogramme ausdrucken kann.

Es ist momentan nicht geplant, graphische Information auszudrucken.














%%% Local Variables: 
%%% mode: latex
%%% TeX-master: "handout1"
%%% End: 
