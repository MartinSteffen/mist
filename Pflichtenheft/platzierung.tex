
\section{Platzierung}
\label{sec:platziering}

\team{\textbf{1:} Daniel Dietrich, Moritz Zahorsky, Christian Buck.
  \textbf{2:} Ralf Th�hle, Paul Mallach}

Die Koordinaten der Transitionssysteme m�ssen berechnet werden.  Dazu mu�
ein \emph{Graphplazierungsalgorithmus} entworfen und implementiert werden.
Die einzelnen Prozesse sowie ihre Zust�nde sollen m�glichst ``sch�n''
dargestellt werden.



\subsubsection*{Schnittstelle}
Gui und Editor. Die Graphplatzierung darf von gescheckter Syntax ausgehen.
Was die Bedeutung der Koordinaten betrifft: siehe den entsprechenden
Abschnitt beim Editor (Abschnitt~\ref{sec:editor}). Da der Editor genau ein
Fenster pro Process bereitstellt, m�ssen nur diese vom Platzierungsgruppe
positioniert werde, nicht ganze Programme. Gui �bernimmt die
Benutzerf�hrung in eigener Regie (nur ein Prozess soll positioniert werden,
z.B., derjenige, dessen Fenster den Fokus hat), oder alle Prozesse sollen
positioniert werden.

Angebote: eine Methode \texttt{position\_process}, die einen Proze� in
abstrakter Syntax nimmt und in mit Koordinaten zur�ckgibt. Ob dies
ebenfalls ein Objekt der abstrakten Syntax ist oder einer anderen
Datenstruktur, wurde noch nicht festgelegt (siehe die Diskussion im
Abschnitt~\ref{sec:editor} des Editors.)

F�r den Anfang sei davon ausgegangen, da� alle Zust�nde \emph{gleich gro�}
seien und Kanten bestenfalls gebogen.


\textbf{Erweiterungsm�glichkeiten:} Zust�nde verschiedener Gr��en,
Ber�cksichtigung der Gr��e der Labels etc.











%%% Local Variables: 
%%% mode: latex
%%% TeX-master: "handout1"
%%% End: 
