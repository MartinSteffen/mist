\section{Einleitung}
\label{sec:einleitung}

Das Dokument beschreibt informell die Funktional�t von \Mist, einem
graphischen Analysetool f�r reaktive Prozesse (Model-checking for
state-transition systems).


Der bereits implementierte \emph{Prototyp} dient als Anregung f�r den
\emph{Editor.} Der Kern der Implementierung, um dem sich alles zu
gruppieren hat, ist die \emph{abstrakte Syntax.} Sie ist (ohne graphische
Komponenten) in Anhang~\ref{sec:AbstrakteSyntax} beschrieben.  Die weiteren
Abschnitte beschreiben Teilaufgaben des Projektes die jeweils als ein
\emph{Paket} implementiert werden. Die \emph{optionalen} Aufgabe sind
sekund�r und dazu gedacht, falls es mehr Gruppen als Aufgaben gibt oder
falls eine Gruppe bereits nach 2 Woche fertig ist; vermutlich ist
beispielsweise die \emph{Check-aufgabe} (Abschnitt~\ref{sec:Checks}) recht
einfach.


Jede der Arbeitsgruppen soll bis zum n�chsten Mal sich �ber die grobe
Aufgabenstellung im Klaren sein, insbesondere �ber  
\begin{itemize}
\item die von ihr bereitgestellten Funktionalit�t und die
\item von ihr von ihr von anderen Gruppen erwarteten Funktionalit�t.
\end{itemize}
Dies gilt vor allem f�r die Gruppe, die die \emph{Integration} �ber die
graphiche Benutzerschnittstelle �bernimmt (Abschnitt~\ref{sec:gui}). Jede
Gruppe soll sich die gestellten Aufgaben anschauen (auch die der anderen),
w�hrend der Woche vorbeikommen und die Aufgabe besprechen. Beim n�chsten
Mal soll dann ein Vertreter der Gruppe seine Vorstellungen erl�utern inkl.\ 
eines \emph{Zeitplanes.}

Da wir fr�h mit der \emph{Integration} beginnen wollen, liegt die
Priorit�t hierbei auf fr�hzeitige Bereitsstellung der versprochenen
Methoden, ohne da� dabei die Funktionalit�t bereits erbracht werden mu�
(als \textit{stubs}).


Von unserer Seite wird eine Implementierung der abstrakten Syntax
(Abschnitt~\ref{sec:AbstrakteSyntax}) geliefert und ein globaler Rahmen
f�r das Projekt (Versionskontrolle etc.)

Falls man aus der Sicht seiner eigenen Gruppe �nderungsw�nsche oder
Erweiterungsw�nsche in Bezug auf die Klassen der abstrakten Syntax hat,
sollte man sie auch, wenn m�glich, bereits n�chste Woche anmelden.









%%% Local Variables: 
%%% mode: latex
%%% TeX-master: "handout1"
%%% End: 
