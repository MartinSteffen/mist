
\section{Simulator}
\label{sec:simulator}


Interaktive Sim ulation ist deren schritt weise Ausf�hrung, soda"s der
Benutzer die Schritte initiier enkann und die Schritte sulta anhand der
Quell-\Mist-Prozesse nachvollziehen kann. Die Funktionalit"at umfa"st
folgende  Punkte:

Benutzerschnittstelle f"ur 

hnittstelle

f"ur einen

oder mehrere

Sc hritte.

Der Ben utzer

soll

einen oder mehrere

nac heinanderfol

gend eSc hritte

initiier en k"onnen.

Sollten Ev ents

vom dem En vironmen

tf "ur

einen

Schritt

not wendig

sein, soll der

Ben utzer

sie eingeb

en k"onnen.

Berec hn ung

des Nac hfolgerzustandes.

Der Algorithm

us zur

Berec

hn ung

des

Nac hfolgezustandes

soll implemen

tiert werden.

Anzeige eines Sc hrittes.

Der vom

Sim ulator

genommene

Schritt muss im Editor angezeigt

werden. Dazu wird die Highligh

t-F unkt

ion des Editors

genutzt. Nic htdeterminism

us. Sollten

mehrere

Nac hfolgerzust

"ande

m"oglic h sein,

soll

der Sim ulator

zw ei M"oglic

hk eiten

zur Verf "ugung

stellen:

1. Der

Ben utzer

wird in den

En tsc heidungsproze

ssein bezogen.

2. Das

To ol entsc

heidet

selbst, welc her Schritt

genommen

werden muss.

Stufe 2. Der

Sim ulator

wird zu einem

Debugger

von Statec

harts erw eitert

.

Dazu m"ussen

folgende

Funktionali

t"at en

hinzugef

"ugt

werden.

Breakp oin ts.

Der Ben utzer

kann Bedingungen

aufstellen, deren Erf "ulltheit

w"ahrend

der Sim ulation

"ub

erpr

"uft

wird.

Dem Ben utzer

wird der Status

der Breakpoin ts mitgeteilt

.

Monitor. Alternativ zur graphisc

hen Darstellung

der Sim ulation,

sollen die Zustands "ub

erg

"ange

als Folgen

von Statuses

dargestellt

werden k"onnen. Ein

Status ist die

laufende

Konfiguration

der Statec

harts und die vorhandenen

Ev ents.

Solc he Folgen

von Statuses

nenn tman

auc hT races.

Sp eic hern

der Traces.

W"ahrend der Sim ulation

aufgetretenen

Traces m"ussen

gesp eic hert

werden

k"onnen.

Abspielen der Traces.

Traces m"ussen

als graphisc

he Sim

ulation

abgespielt

werden k"onnen.



%%% Local Variables: 
%%% mode: latex
%%% TeX-master: "handout1"
%%% End: 
