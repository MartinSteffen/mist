\section{Simulator}
\label{sec:simulator}

\team{Michael Goemann, Michael Nimser}



Interaktive Simulation eines Programmes ist deren schrittweise Ausf�hrung,
soda� der Benutzer die Schritte initiieren und sie anhand der
Quell-\Mist-Prozesse nachvollziehen kann.  Der Simulator realisiert die
\emph{Semantik} aus Anhang~\ref{sec:semantik}.

Die Funktionalit�t umfa�t folgende Punkte:

\paragraph{Berechnung des Nachfolgezustandes:}

Der Algorithmus zur Berechnung des Nachfolgezustandes soll implementiert
werden. Sollte externe Kommunikation f�r einen Schritt notwendig sein, soll
der Benutzer sie eingeben k�nnen.

\paragraph{Anzeige eines Schrittes:}
Der vom Simulator genommene Schritt mu� im Editor angezeigt werden. Dazu
wird die Highlight-Funktion des Editors genutzt.


\paragraph{Nichtdeterminismus:}

Sollten mehrere Nachfolgerzust�nde m�glich sein, soll der Simulator zwei
M�glichkeiten zur Verf�gung stellen:
\begin{enumerate}
\item Der Benutzer wird in den Entscheidungsproze� einbezogen.
\item Der Simulator  entscheidet, welcher Schritt genommen wird.
\end{enumerate}

F�r die erste M�glichkeit mu� man zwischen externen und internen
Kommunikationen entscheiden. Wie machen es auf die einfachst-m�gliche Art:
jedes Label, welches \emph{sowohl} als Eingabe als auch als Ausgabe label
vorkommt, sei ein \emph{internes Label,} alle anderen \emph{extern.}

Der letzte Punkt |Unterscheiden zwischen externer und interner
Kommunikation |kann in einer ersten Stufe der Implementierung noch
unber�cksichtigt bleiben.

Weiteres f�r erweiterte Funktionalit�t, was in der ersten Stufe
unber�cksichtigt bleibt:
\begin{itemize}
\item Back-stepping
\item Aufzeichnen (und Speichern) der genommenen Schritte.
\end{itemize}




%%% Local Variables: 
%%% mode: latex
%%% TeX-master: "handout1"
%%% End: 
