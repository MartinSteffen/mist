\section{Einleitung}
\label{sec:einleitung}

Das Dokument beschreibt informell die Funktionalit�t von \Mist, einem
graphischen Analysetool f�r reaktive Prozesse (\textit{Model-Checking for
  state-transition systems}).


Der bereits implementierte \emph{Prototyp} dient als Anregung f�r den
\emph{Editor.} Der Kern der Implementierung, um den sich alles zu
gruppieren hat, ist die \emph{abstrakte Syntax.} Sie ist (ohne graphische
Komponenten) in Anhang~\ref{sec:AbstrakteSyntax} beschrieben.  Die weiteren
Abschnitte beschreiben Teilaufgaben des Projektes die jeweils als ein
\emph{Paket} implementiert werden. Die \emph{optionalen} Aufgabe sind
sekund�r und dazu gedacht, falls es mehr Gruppen als Aufgaben gibt oder
falls eine Gruppe bereits nach 2 Woche fertig ist; vermutlich ist
beispielsweise die \emph{Check-Aufgabe} (Abschnitt~\ref{sec:checks}) recht
einfach.

Insbesondere legt das Dokument f�r jedes Paktet
\begin{itemize}
\item die von ihr bereitgestellte Funktionalit�t, und die
\item die  von den anderen Gruppen erwartete Funktionalit�t fest.
\end{itemize}
Dies gilt vor allem f�r die Gruppe, die die \emph{Integration} �ber die
graphische Benutzerschnittstelle �bernimmt (Abschnitt~\ref{sec:gui}).

Da wir fr�h mit der \emph{Integration} beginnen wollen, liegt die Priorit�t
hierbei auf fr�hzeitiger Bereitsstellung der versprochenen Methoden, ohne
da� dabei die Funktionalit�t bereits erbracht werden mu� (als
\textit{stubs}).


Von unserer Seite wird eine Implementierung der abstrakten Syntax
(Abschnitt~\ref{sec:AbstrakteSyntax}) geliefert und ein globaler Rahmen
f�r das Projekt (Versionskontrolle etc.).

Falls man aus der Sicht seiner eigenen Gruppe �nderungs- oder
Erweiterungsw�nsche in Bezug auf die Klassen der abstrakten Syntax hat,
sollte man sie auch sobald wie m�glich, anmelden, bzw.\ nach passender
Warnung an alle, es selber anpassen.


%%% Local Variables: 
%%% mode: latex
%%% ispell-dictionary: "deutsch"
%%% TeX-master: "handout1"
%%% End: 
